\documentclass[11pt,a4paper]{extarticle}
\usepackage{geometry}
 \geometry{
 a4paper,
 total={160mm,237mm},
 left=25mm,
 top=25mm,
 }
\usepackage[utf8x]{inputenc}
\usepackage[english]{babel}
\usepackage{bbm}
\usepackage[usenames]{color}
\usepackage{hyperref}
\usepackage{colortbl}
\usepackage{amsmath}
\usepackage{amssymb}
\usepackage[pdftex]{graphicx}
\usepackage{amsthm}
\usepackage{caption}
\usepackage[ruled,vlined]{algorithm2e}
\usepackage{authblk}

 
\title{Course project \\
"Risk analysis and high-dimensional integrals"}
\author{}

\begin{document}

\maketitle

\section{Team members}

\begin{itemize}
    
    \item Igor Silin (leader)
    
    \item Nikita Puchkin
    
    \item Aleksandr Podkopaev
    
\end{itemize}

\section{Background}

Many stock prices can modelled using Brownian motion, i.e. the next state depends on the previous state plus some random event. Important statistical options (for example, option calls) can be then estimated as expectations of random process. The straightforward way is to do Monte-Carlo simulation, but it can not give high accuracy. Another way is to write down the expectation as a path integral and then approximate it by a multi-dimensional integral. 

For similar integrals exists a new technique described in \href{http://arxiv.org/abs/1504.06149}{[1]}. The idea of the project is to test it for examples from risk analysis.

\section{Problem formulation}

	Problem statement is described in  \href{http://www.columbia.edu/~ks20/FE-Notes/4700-07-Notes-GBM.pdf}{[2]}. Let $f(t, x)$ denote the price at time $t$ of a derivative of stock (such as a European call option) when $S(t) = x$. Then $f$ must satisfy the partial differential equation: 

\[
	\frac{\partial f}{\partial t} + rx \frac{\partial f}{\partial x} + \frac12 \sigma^2 x^2 \frac{\partial^2 f}{\partial x^2} = rf.
\]
This is called Black-Scholes partial differential equation.

Consider variable substitution of the following type: $x=e^u$. Then after sequence of elementary transformations one obtains:

\[
	\frac{\partial f}{\partial t} + \left(r-\frac{\sigma^2}{2}\right) \frac{\partial f}{\partial u} + \frac{\sigma^2}{2} \frac{\partial^2 f}{\partial u^2} = rf.
\]
Different initial conditions corresponds to different financial derivatives. This equation is similar to one from \href{http://arxiv.org/abs/1504.06149}{[1]}, where the method for solving such PDE is proposed.

\section{Data}
	
Currently, our team is looking for appropriate financial data which both MC and the proposed approach can be applied for.
					 							
\section{Scope}

As a result we consider implemented algorithm for calculating the integral and the error given parameters of the model and discretization parameters.	We are also going to provide comparison with other existing methods. 

Our work will be divided in the following stages:
\begin{itemize}
    \item Finding, reading and understanding of papers on high-dimensional integrals in risk analysis and partial differential equations in mathematical finance in general.
    \item Understanding the method described in \href{http://arxiv.org/abs/1504.06149}{[1]} in details.
    \item Implementing the algorithm on pyhton.
    \item Finding appropriate financial data for testing.
    \item Evaluation of the results, comparison with other methods.
\end{itemize}

\section{Evaluation}

Evaluation of performance can be done in the following ways:

\begin{itemize}
    
    \item Based on previous research papers, we will compare the overall quality and execution time with presented there.
    
    \item If we find no appropriate paper with precisely described data and algorithm performance, we will implement MC-based method ourselves.
    
\end{itemize}


\section{References}
     $[1]$  A low-rank approach to the computation of path integrals. M.S. Litsarev, I.V. Oseledets, 2015.\\
    $[2]$ Karl Sigman notes, 2006.\\

    


			
		

\end{document}
