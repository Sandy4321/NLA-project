\documentclass[11pt,a4paper]{extarticle}
\usepackage{geometry}
 \geometry{
 a4paper,
 total={160mm,237mm},
 left=25mm,
 top=25mm,
 }
\usepackage[utf8x]{inputenc}
\usepackage[english]{babel}
\usepackage{bbm}
\usepackage[usenames]{color}
\usepackage{hyperref}
\usepackage{colortbl}
\usepackage{amsmath}
\usepackage{amssymb}
\usepackage[pdftex]{graphicx}
\usepackage{amsthm}
\usepackage{caption}
\usepackage[ruled,vlined]{algorithm2e}
\usepackage{authblk}

 
\title{NLA course project \\
"Risk analysis and high-dimensional integrals"}
\author{Igor Silin, Nikita Puchkin, Aleksandr Podkopaev}

\begin{document}

\maketitle

\section{Problem formulation}

	Problem statement is described in  \href{http://www.columbia.edu/~ks20/FE-Notes/4700-07-Notes-GBM.pdf}{[2]}. Let $c(s, t)$ denote the price at time $t$ of a derivative of a stock (such as a European call option) when the price of the stock $S(t) = s$. Then $c$ must satisfy the partial differential equation: 

\begin{equation}
	\begin{aligned}
		&\frac{\partial c(s,t)}{\partial t} + rs \frac{\partial c(s,t)}{\partial s} + \frac12 \gamma^2 s^2 \frac{\partial^2 c(s,t)}{\partial s^2} = rc(s,t),
		\;\;\;s \in \mathbbm{R}_{+},\;\; t \in [0;\;T^{\prime}],\\
		&c(s,T^{\prime}) = g(s), \;\;\; s \in \mathbbm{R}_{+},\\
		&c(0,t) = 0, \;\;\; t \in [0;\;T^{\prime}].
	\end{aligned}
\end{equation}
Here  $r$ is the risk-free interest rate and $\gamma$ is the volatility of the stock.
This is called Black-Scholes partial differential equation. Different terminal conditions $g(s)$ corresponds to different types of derivatives (e.g. European call option).

Consider variable substitution of the following type: $x=\ln s$ and $w(x,t) = c(s, t)$. Then after sequence of elementary transformations one obtains:
\begin{equation}
	\begin{aligned}
		&\frac{\partial w(x,t)}{\partial t} + \left(r-\frac{\gamma^2}{2}\right) \frac{\partial w(x,t)}{\partial x} + 
		\frac12 \gamma^2 \frac{\partial^2 w(x,t)}{\partial x^2} = rw(x,t),
		\;\;\;x \in \mathbbm{R},\;\; t \in [0;\;T^{\prime}],\\
		&w(x,T^{\prime}) = g(e^x), \;\;\; x \in \mathbbm{R},\\
		&w(-\infty,t) = 0, \;\;\; t \in [0;\;T^{\prime}].
	\end{aligned}
\end{equation}
To get rid of the drift term $\left(r-\frac{\gamma^2}{2}\right) \frac{\partial w(x,t)}{\partial x}$ we introduce
\begin{equation}
	\begin{aligned}
		v(x,t) = e^{\frac{1}{\gamma^2} (r-\frac{\gamma^2}{2})x} w(x,t),
	\end{aligned}
\end{equation}
and obtain
\begin{equation}
	\begin{aligned}
		&\frac{\partial v(x,t)}{\partial t} + \frac12 \gamma^2 \frac{\partial^2 v(x,t)}{\partial x^2} = 
		\left( r +  \frac{1}{2\gamma^2} \left( r-\frac{\gamma^2}{2}\right)^2 \right) v(x,t),
	 	\;\;\;x \in \mathbbm{R},\;\; t \in [0;\;T^{\prime}],\\
		&v(x,T^{\prime}) = e^{\frac{1}{\gamma^2} (r-\frac{\gamma^2}{2})x} g(e^x), \;\;\; x \in \mathbbm{R},\\
		&v(-\infty,t) = 0, \;\;\; t \in [0;\;T^{\prime}].
	\end{aligned}
\end{equation}
Introducing 
$\sigma = \frac12 \gamma^2$, $V(x,t) = V = r +  \frac{1}{2\gamma^2} \left( r-\frac{\gamma^2}{2}\right)^2$ and
$f(x) = e^{\frac{1}{\gamma^2} (r-\frac{\gamma^2}{2})x} g(e^x)$  we have
\begin{equation}
	\begin{aligned}
		&-\frac{\partial v(x,t)}{\partial t} = \sigma \frac{\partial^2 v(x,t)}{\partial x^2} - V(x,t) v(x,t),
		\;\;\;x \in \mathbbm{R},\;\; t \in [0;\;T^{\prime}],\\
		&v(x,T^{\prime}) = f(x), \;\;\; x \in \mathbbm{R},\\
		&v(-\infty,t) = 0, \;\;\; t \in [0;\;T^{\prime}].
	\end{aligned}
\end{equation}


Finally, we set $u(x,t) = v(x, T^{\prime}-t)$ and obtain
\begin{equation}
	\begin{aligned}
		&\frac{\partial u(x,t)}{\partial t} = \sigma \frac{\partial^2 u(x,t)}{\partial x^2} - V(x,t) u(x,t),
		\;\;\;x \in \mathbbm{R},\;\; t \in [0;\;T^{\prime}],\\
		&v(x,0) = f(x), \;\;\; x \in \mathbbm{R},\\
		&v(-\infty,t) = 0, \;\;\; t \in [0;\;T^{\prime}].
	\end{aligned}
\end{equation}
The condition $v(-\infty,t) = 0$ will be satisfied automatically (prove!), so our partial differential equation is
\begin{equation}
	\begin{aligned}
		&\frac{\partial u(x,t)}{\partial t} = \sigma \frac{\partial^2 u(x,t)}{\partial x^2} - V(x,t) u(x,t),
		\;\;\;x \in \mathbbm{R},\;\; t \in [0;\;T^{\prime}],\\
		&v(x,0) = f(x), \;\;\; x \in \mathbbm{R}.
	\label{OseledetsForm}
	\end{aligned}
\end{equation}
This form is exactly one that is considered in Oseledets's paper.

So, when we have $g(s)$, $\gamma$, $r$, $T^{\prime}$ from our financial model, we just set
\begin{equation}
	\begin{aligned}
		f(x) = e^{\frac{1}{\gamma^2} (r-\frac{\gamma^2}{2})x} g(e^x), \;\; \sigma = \frac12 \gamma^2, \;\;\; V(x,t) = V = r +  \frac{1}{2\gamma^2} \left( r-\frac{\gamma^2}{2}\right)^2,
	\end{aligned}
\end{equation}
solve (\ref{OseledetsForm}) with this parameters, and reconstruct the desired function $c(s,t)$ as
\begin{equation}
	\begin{aligned}
		c(s,t) = w(e^x, t) = e^{-\frac{1}{\gamma^2} (r-\frac{\gamma^2}{2})\ln s} v(\ln s, t) = e^{-\frac{1}{\gamma^2} (r-\frac{\gamma^2}{2})\ln s} u(\ln s, T^{\prime} - t).
	\end{aligned}
\end{equation}

		

\end{document}
